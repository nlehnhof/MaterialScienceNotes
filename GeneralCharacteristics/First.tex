\documentclass[12pt]{article}

% --- Packages ---
\usepackage{amsmath}
\usepackage{amssymb}
\usepackage{siunitx}
\usepackage{physics}
\usepackage{geometry}
\geometry{margin=1in}

% --- Title Info ---
\title{General Characteristics of Materials: First Principles}
\author{Nate}
\date{\today}

\begin{document}
\maketitle
\tableofcontents
\newpage

\section{Introduction}
First week of the course focuses on the fundamental characteristics of materials, including their mechanical properties and how they respond to various forces.

\section{Materials}
\subsection{Ceramics}
Salts -- ionic bonding, brittle, high melting points, poor conductors of electricity and heat.
Ex. alumina (Al$_2$O$_3$), silicon carbide (SiC), and zirconia (ZrO$_2$).

\begin{list}{ceramics}{spacing}
    \item 
\end{list}

\subsection{Metals}
Metalic bonding is strong. Metals are ductile, malleable, and good conductors of electricity and heat.
They typically have high melting points and are often used in structural applications.
Ex. aluminum, copper, iron, and steel.

\begin{list}{metals}{spacing}
    \item Metalic Bonding
    \item Variable Bond Energy
    \item moderate $T_m$
    \item moderate E
    \item moderate $\alpha$
\end{list}

\subsection{Polymers}
Polymers are made up of long chains of repeating molecular units. 
They can be flexible or rigid, and their properties can vary widely depending on their chemical structure and the way they are processed. 
Polymers are generally poor conductors of heat and electricity, and they can be engineered to have specific mechanical properties such as toughness or elasticity.
Ex. polyethylene, polystyrene, and nylon.

Polymers have a small melting point because they have weak bonds.

\begin{list}{Polymers}{spacing}
    \item Covalent and Secondary Bonds
    \item Secondary bonding dominates
    \item Directional properties
    \item small $T_m$
    \item small E
    \item large $\alpha$
\end{list}

\subsection{Composites}
Composites are materials made from two or more constituent materials with significantly different physical or chemical properties. 
When combined, they produce a material with characteristics different from the individual components. 
Ex. wood (cellulose fibers in a lignin matrix), fiberglass (glass fibers in a polymer matrix).

\section{Bonding Types}
\subsection{Primary Bonds}
Primary bonding: ionic - transfer of electrons, covalent - sharing of electrons, metallic bond - delocalized cloud of electrons,  
\[
\% \text{ Ionic Character } = (1 - e^{-\frac{(\chi_A - \chi_B)^2}{4}}) \times 100\%
\]

where $\chi_A$ and $\chi_B$ are the electronegativities of atoms A and B, respectively.

\subsection{Secondary Bonds}
Van der Walls forces, dipole-dipole interactions, hydrogen bonding are caused by the attraction/repulsion of charged particles or dipoles in molecules.
These forces are generally weaker than primary bonds but can significantly influence the physical properties of materials, such as melting point and solubility.

\section{Energy of the Bonds}
Energy = Force * Distance.

\[ E_N = E_A + E_R = - \frac{A}{r} + \frac{B}{r^n}\]
n is the Leanord Jones Potential and is generally 8 or 9.
r is the distance between atoms.
$E_N$ is the Net Energy.
$r_0$ is the equilibrium distance between the two atoms.

To find $r_0$, find the $E_N$ equation derivative and set it equal to 0. Then, plug $r_0$ back in to the original equation to get $E_0$.

*IMPORTANT SUMMARY*
$T_m$ is larger if $E_0$ is larger.
$\alpha$ is larger if E0 is smaller / well-size.
$\alpha$ is the coefficient of thermal expansion.
E is larger whichever is steepest.

Ex. Test Question: *Van der Waals bonds most commonly occur in polymers and result in low melting temperatures.*

\section{Equations}
Stress: $\sigma = \frac{F}{A}$

\end{document}